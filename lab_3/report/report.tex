\documentclass[a4paper,12pt]{extarticle}
\usepackage[utf8x]{inputenc}
\usepackage[T1,T2A]{fontenc}
\usepackage[russian]{babel}
\usepackage{hyperref}
\usepackage{indentfirst}
\usepackage{listings}
\usepackage{color}
\usepackage{here}
\usepackage{array}
\usepackage{multirow}
\usepackage{graphicx}
\usepackage{amsmath}
%\usepackage[dvips]{graphicx}

\usepackage{caption}
\renewcommand{\lstlistingname}{Программа} % заголовок листингов кода

\bibliographystyle{ugost2008ls}

\lstset{ %
extendedchars=\true,
keepspaces=true,
language=C,						% choose the language of the code
basicstyle=\footnotesize,		% the size of the fonts that are used for the code
numbers=left,					% where to put the line-numbers
numberstyle=\footnotesize,		% the size of the fonts that are used for the line-numbers
stepnumber=1,					% the step between two line-numbers. If it is 1 each line will be numbered
numbersep=5pt,					% how far the line-numbers are from the code
backgroundcolor=\color{white},	% choose the background color. You must add \usepackage{color}
showspaces=false				% show spaces adding particular underscores
showstringspaces=false,			% underline spaces within strings
showtabs=false,					% show tabs within strings adding particular underscores
frame=single,           		% adds a frame around the code
tabsize=2,						% sets default tabsize to 2 spaces
captionpos=t,					% sets the caption-position to top
breaklines=true,				% sets automatic line breaking
breakatwhitespace=false,		% sets if automatic breaks should only happen at whitespace
escapeinside={\%*}{*)},			% if you want to add a comment within your code
postbreak=\raisebox{0ex}[0ex][0ex]{\ensuremath{\color{red}\hookrightarrow\space}},
texcl=true,
%inputpath=listings,                     % директория с листингами
}

\usepackage[left=2cm,right=2cm,
top=2cm,bottom=2cm,bindingoffset=0cm]{geometry}

%% Нумерация картинок по секциям
\usepackage{chngcntr}
\counterwithin{figure}{section}
\counterwithin{table}{section}

%%Точки нумерации заголовков
\usepackage{titlesec}
\titlelabel{\thetitle.\quad}
\usepackage[dotinlabels]{titletoc}

%% Оформления подписи рисунка
\addto\captionsrussian{\renewcommand{\figurename}{Рисунок}}
\captionsetup[figure]{labelsep = period}

%% Подпись таблицы
\DeclareCaptionFormat{hfillstart}{\hfill#1#2#3\par}
\captionsetup[table]{format=hfillstart,labelsep=newline,justification=centering,skip=-10pt,textfont=bf}

%% Путь к каталогу с рисунками
%\graphicspath{{fig/}}

%% Внесение titlepage в учёт счётчика страниц
\makeatletter
\renewenvironment{titlepage} {
 \thispagestyle{empty}
}
\makeatother

\begin{document}	% начало документа

% Титульная страница
\begin{titlepage}	% начало титульной страницы

	\begin{center}		% выравнивание по центру

		\large Санкт-Петербургский политехнический университет Петра Великого\\
		\large Институт компьютерных наук и технологий \\
		\large Высшая школа интеллектуальных систем и суперкомпьютерных технологий\\[6cm]
		% название института, затем отступ 6см

		\huge Алгоритмы цифровой обработки изображений\\[0.5cm] % название работы, затем отступ 0,5см
		\large Отчет по лабораторной работе №3 \\[0.1cm]
		\large Алгоритмы сглаживания изображений \\[5cm]

	\end{center}


	\begin{flushright} % выравнивание по правому краю
		\begin{minipage}{0.25\textwidth} % врезка в половину ширины текста
			\begin{flushleft} % выровнять её содержимое по левому краю

				\large\textbf{Работу выполнил:}\\
				\large Медведев А.В.\\
				\large {Группа:} 3540901/01502\\

				\large \textbf{Преподаватель:}\\
				\large Абрамов Н.А.

			\end{flushleft}
		\end{minipage}
	\end{flushright}

	\vfill % заполнить всё доступное ниже пространство

	\begin{center}
	\large Санкт-Петербург\\
	\large \the\year % вывести дату
	\end{center} % закончить выравнивание по центру

\end{titlepage} % конец титульной страницы

\vfill % заполнить всё доступное ниже пространство

% Содержание
% Содержание
\renewcommand\contentsname{\centerline{Содержание}}
\tableofcontents
\newpage


\section{Цель работы}

Ознакомиться со способами сглаживания изображений.


\section{Ход выполнения работы}


\subsection{Наложение шума}

\begin{figure}[H]
	\begin{center}
		\includegraphics[scale=1.0]{../noise/out/1/resource_img}
		\caption{Исходное изображение }
		%		\label{pic:source_pic_1} % название для ссылок внутри кода
	\end{center}
\end{figure}

\begin{figure}[H]
	\begin{center}
		\includegraphics[scale=1.0]{../noise/out/1/noise_img}
		\caption{Изображение с наложенным шумом}
		%		\label{pic:source_pic_1} % название для ссылок внутри кода
	\end{center}
\end{figure}

\subsection{Билотеральный метод}

Фильтр исполльзует Гаусово распределение учитывая пространственную разность и разность интенсивностей.

Формула для билатерального фильтра:

$BF[I]_p = \frac{1}{W_p} \sum\limits_{q \in S} G_{\sigma_s} (||p-q||) G_{\sigma_r} (|I_p-I_q|) I_q$

Ниже представлены результаты применения различных значений сигм (пространственной и интенсивностей):

\begin{figure}[H]
	\begin{minipage}[h]{0.49\linewidth}
		\center{\includegraphics[width=1\linewidth]{../bilaterally/out/1/resource_img} \\ Изображение с шумом}
	\end{minipage}
	\hfill
	\begin{minipage}[h]{0.49\linewidth}
		\center{\includegraphics[width=1\linewidth]{../bilaterally/out/111/bilateral_img_2_16} \\ Sigma = 2}
	\end{minipage}
	\vfill
	\begin{minipage}[h]{0.49\linewidth}
		\center{\includegraphics[width=1\linewidth]{../bilaterally/out/111/bilateral_img_4_16} \\ Sigma = 4}
	\end{minipage}
	\hfill
	\begin{minipage}[h]{0.49\linewidth}
		\center{\includegraphics[width=1\linewidth]{../bilaterally/out/111/bilateral_img_8_16} \\ Sigma = 6}
	\end{minipage}
	\vfill
	\begin{minipage}[h]{0.49\linewidth}
		\center{\includegraphics[width=1\linewidth]{../bilaterally/out/111/bilateral_img_10_16} \\ Sigma = 8}
	\end{minipage}
	\hfill
	\begin{minipage}[h]{0.49\linewidth}
		\center{\includegraphics[width=1\linewidth]{../bilaterally/out/111/bilateral_img_16_16} \\ Sigma = 10}
	\end{minipage}	
	\caption{$\sigma интенсивности = 16$}
\end{figure}

\begin{figure}[H]
	\begin{minipage}[h]{0.49\linewidth}
		\center{\includegraphics[width=1\linewidth]{../bilaterally/out/1/resource_img} \\ Изображение с шумом}
	\end{minipage}
	\hfill
	\begin{minipage}[h]{0.49\linewidth}
		\center{\includegraphics[width=1\linewidth]{../bilaterally/out/111/bilateral_img_2_24} \\ Sigma = 2}
	\end{minipage}
	\vfill
	\begin{minipage}[h]{0.49\linewidth}
		\center{\includegraphics[width=1\linewidth]{../bilaterally/out/111/bilateral_img_4_24} \\ Sigma = 4}
	\end{minipage}
	\hfill
	\begin{minipage}[h]{0.49\linewidth}
		\center{\includegraphics[width=1\linewidth]{../bilaterally/out/111/bilateral_img_8_24} \\ Sigma = 6}
	\end{minipage}
	\vfill
	\begin{minipage}[h]{0.49\linewidth}
		\center{\includegraphics[width=1\linewidth]{../bilaterally/out/111/bilateral_img_10_24} \\ Sigma = 8}
	\end{minipage}
	\hfill
	\begin{minipage}[h]{0.49\linewidth}
		\center{\includegraphics[width=1\linewidth]{../bilaterally/out/111/bilateral_img_16_24} \\ Sigma = 10}
	\end{minipage}	
	\caption{$\sigma интенсивности = 24$}
\end{figure}

\begin{figure}[H]
	\begin{minipage}[h]{0.49\linewidth}
		\center{\includegraphics[width=1\linewidth]{../bilaterally/out/1/resource_img} \\ Изображение с шумом}
	\end{minipage}
	\hfill
	\begin{minipage}[h]{0.49\linewidth}
		\center{\includegraphics[width=1\linewidth]{../bilaterally/out/111/bilateral_img_2_32} \\ Sigma = 2}
	\end{minipage}
	\vfill
	\begin{minipage}[h]{0.49\linewidth}
		\center{\includegraphics[width=1\linewidth]{../bilaterally/out/111/bilateral_img_4_32} \\ Sigma = 4}
	\end{minipage}
	\hfill
	\begin{minipage}[h]{0.49\linewidth}
		\center{\includegraphics[width=1\linewidth]{../bilaterally/out/111/bilateral_img_8_32} \\ Sigma = 6}
	\end{minipage}
	\vfill
	\begin{minipage}[h]{0.49\linewidth}
		\center{\includegraphics[width=1\linewidth]{../bilaterally/out/111/bilateral_img_10_32} \\ Sigma = 8}
	\end{minipage}
	\hfill
	\begin{minipage}[h]{0.49\linewidth}
		\center{\includegraphics[width=1\linewidth]{../bilaterally/out/111/bilateral_img_16_32} \\ Sigma = 10}
	\end{minipage}	
	\caption{$\sigma интенсивности = 32$}
\end{figure}

\begin{figure}[H]
	\begin{minipage}[h]{0.49\linewidth}
		\center{\includegraphics[width=1\linewidth]{../bilaterally/out/1/resource_img} \\ Изображение с шумом}
	\end{minipage}
	\hfill
	\begin{minipage}[h]{0.49\linewidth}
		\center{\includegraphics[width=1\linewidth]{../bilaterally/out/111/bilateral_img_2_64} \\ Sigma = 2}
	\end{minipage}
	\vfill
	\begin{minipage}[h]{0.49\linewidth}
		\center{\includegraphics[width=1\linewidth]{../bilaterally/out/111/bilateral_img_4_64} \\ Sigma = 4}
	\end{minipage}
	\hfill
	\begin{minipage}[h]{0.49\linewidth}
		\center{\includegraphics[width=1\linewidth]{../bilaterally/out/111/bilateral_img_8_64} \\ Sigma = 6}
	\end{minipage}
	\vfill
	\begin{minipage}[h]{0.49\linewidth}
		\center{\includegraphics[width=1\linewidth]{../bilaterally/out/111/bilateral_img_10_64} \\ Sigma = 8}
	\end{minipage}
	\hfill
	\begin{minipage}[h]{0.49\linewidth}
		\center{\includegraphics[width=1\linewidth]{../bilaterally/out/111/bilateral_img_16_64} \\ Sigma = 10}
	\end{minipage}	
	\caption{$\sigma интенсивности = 64$}
\end{figure}

\begin{figure}[H]
	\begin{minipage}[h]{0.49\linewidth}
		\center{\includegraphics[width=1\linewidth]{../bilaterally/out/1/resource_img} \\ Изображение с шумом}
	\end{minipage}
	\hfill
	\begin{minipage}[h]{0.49\linewidth}
		\center{\includegraphics[width=1\linewidth]{../bilaterally/out/111/bilateral_img_2_96} \\ Sigma = 2}
	\end{minipage}
	\vfill
	\begin{minipage}[h]{0.49\linewidth}
		\center{\includegraphics[width=1\linewidth]{../bilaterally/out/111/bilateral_img_4_96} \\ Sigma = 4}
	\end{minipage}
	\hfill
	\begin{minipage}[h]{0.49\linewidth}
		\center{\includegraphics[width=1\linewidth]{../bilaterally/out/111/bilateral_img_8_96} \\ Sigma = 6}
	\end{minipage}
	\vfill
	\begin{minipage}[h]{0.49\linewidth}
		\center{\includegraphics[width=1\linewidth]{../bilaterally/out/111/bilateral_img_10_96} \\ Sigma = 8}
	\end{minipage}
	\hfill
	\begin{minipage}[h]{0.49\linewidth}
		\center{\includegraphics[width=1\linewidth]{../bilaterally/out/111/bilateral_img_16_96} \\ Sigma = 10}
	\end{minipage}	
	\caption{$\sigma интенсивности = 96$}
\end{figure}

\begin{table}[H]
	\caption{Сравнение модулей разности полученных изображений}
	\label{tabular:timesandtenses}
	\begin{center}
		\begin{tabular}{ccc}
			Пространственая сигма & Сигма итенсивности & Модуль разности \\
			2 & 16 & 14.75439 \\
			2 & 24 & 14.75462 \\
			2 & 32 & 14.75463 \\
			2 & 64 & 14.75466 \\
			2 & 96 & 14.75465 \\
			4 & 16 & 14.79203 \\
			4 & 24 & 14.79222 \\
			4 & 32 & 14.79230 \\
			4 & 64 & 14.79238 \\
			4 & 96 & 14.79240 \\
			8 & 16 & 14.92022 \\
			8 & 24 & 14.92071 \\
			8 & 32 & 14.92090 \\
			8 & 64 & 14.92113 \\
			8 & 96 & 14.92117 \\
			10 & 16 & 15.00214 \\
			10 & 24 & 15.00265 \\
			10 & 32 & 15.00268 \\
			10 & 64 & 15.00292 \\
			10 & 96 & 15.00298 \\
			16 & 16 & 15.31712 \\
			16 & 24 & 15.31827 \\
			16 & 32 & 15.31879 \\
			16 & 64 & 15.31928 \\
			16 & 96 & 15.31930 \\
		\end{tabular}
	\end{center}
\end{table}

Вывод:

Лучший показатель наблюдется при следующих показателях: пространственная $\sigma$ = 16, $\sigma$ интенсивности = 2.


\subsection{Метод Гауса}

Первым шагом является нахождение ядра гауса по следующей формуле:

G(x,y) = $\frac{1}{2 \pi \sigma^2} e^{-\frac{x^2+y^2}{2\sigma^2}}$

Затем используется операция всёртки для сглаживания изображения:

G[i,j] = $\sum\limits_{u=-k}^k \sum\limits_{v=-k}^k H[u,v]F[i+u,j+v]$

Формула операции сглаживания Гауссовым фильтром:

$GB[I]_p = \sum\limits_{q \in S} G_\sigma (||p-q||)I_q$

Ниже представлены результаты применения различных значений параметра $\sigma$:
\begin{figure}[H]
	\begin{minipage}[h]{0.49\linewidth}
		\center{\includegraphics[width=1\linewidth]{../gaus/out/1/resource_img} \\ Изображение с шумом}
	\end{minipage}
	\hfill
	\begin{minipage}[h]{0.49\linewidth}
		\center{\includegraphics[width=1\linewidth]{../gaus/out/1/gauss_img_sigma_2} \\ Sigma = 2}
	\end{minipage}
	\vfill
	\begin{minipage}[h]{0.49\linewidth}
		\center{\includegraphics[width=1\linewidth]{../gaus/out/1/gauss_img_sigma_4} \\ Sigma = 4}
	\end{minipage}
	\hfill
	\begin{minipage}[h]{0.49\linewidth}
		\center{\includegraphics[width=1\linewidth]{../gaus/out/1/gauss_img_sigma_6} \\ Sigma = 6}
	\end{minipage}
	\vfill
	\begin{minipage}[h]{0.49\linewidth}
		\center{\includegraphics[width=1\linewidth]{../gaus/out/1/gauss_img_sigma_8} \\ Sigma = 8}
	\end{minipage}
	\hfill
	\begin{minipage}[h]{0.49\linewidth}
		\center{\includegraphics[width=1\linewidth]{../gaus/out/1/gauss_img_sigma_10} \\ Sigma = 10}
	\end{minipage}	
	\vfill
	\begin{minipage}[h]{0.49\linewidth}
		\center{\includegraphics[width=1\linewidth]{../gaus/out/1/gauss_img_sigma_12} \\ Sigma = 12}
	\end{minipage}
	\hfill
	\begin{minipage}[h]{0.49\linewidth}
		\center{\includegraphics[width=1\linewidth]{../gaus/out/1/gauss_img_sigma_14} \\ Sigma = 14}
	\end{minipage}	
\end{figure}

\begin{table}[H]
	\caption{Сравнение модулей разности полученных изображений}
	\label{tabular:timesandtenses}
	\begin{center}
		\begin{tabular}{ccc}
			Сигма итенсивности & Модуль разности \\
			2 & 18.5538 \\
			4 & 19.3089 \\
			6 & 19.4719 \\
			8 & 19.5307 \\
			10 & 19.5575 \\
			12 & 19.5719 \\
			14 & 19.5809 \\
			16 & 19.5865 \\
		\end{tabular}
	\end{center}
\end{table}

Вывод:

Наилучший результат даёт значение $\sigma$ равное 2. 

\subsection{Метод нелокальных средних}

Данный фильтр использует Гауссово распределение с учётом разности расстояния до похожих участков изображения.\\

Формула для фильтра нелокальными средними:

$I_d(j) = \sum\limits_{j \in I} \omega (i,j) I(j)$

Формула для веса фильтра:
$\omega(i,j) = \frac{1}{Z(i)} e^{- \frac{||I(N_i)-I(N_j)||}{h^2}}$

Формула нормирующего делителя фильтра:
$Z(i) = \sum\limits_{j \in I} e^{- \frac{||I(N_i)-I(N_j)||}{h^2}}$

Ниже представлены результаты применения различных значений параметра $\sigma$:

\begin{figure}[H]
	\begin{minipage}[h]{0.49\linewidth}
		\center{\includegraphics[width=1\linewidth]{../nlm/out/1/resource_img} \\ Изображение с шумом}
	\end{minipage}
	\hfill
	\begin{minipage}[h]{0.49\linewidth}
		\center{\includegraphics[width=1\linewidth]{../nlm/out/1/nlm_img_10} \\ Sigma = 10}
	\end{minipage}
	\vfill
	\begin{minipage}[h]{0.49\linewidth}
		\center{\includegraphics[width=1\linewidth]{../nlm/out/1/nlm_img_20} \\ Sigma = 20}
	\end{minipage}
	\hfill
	\begin{minipage}[h]{0.49\linewidth}
		\center{\includegraphics[width=1\linewidth]{../nlm/out/1/nlm_img_30} \\ Sigma = 30}
	\end{minipage}
	\vfill
	\begin{minipage}[h]{0.49\linewidth}
		\center{\includegraphics[width=1\linewidth]{../nlm/out/1/nlm_img_40} \\ Sigma = 40}
	\end{minipage}
	\hfill
	\begin{minipage}[h]{0.49\linewidth}
		\center{\includegraphics[width=1\linewidth]{../nlm/out/1/nlm_img_60} \\ Sigma = 60}
	\end{minipage}
\end{figure}

\begin{table}[H]
	\caption{Сравнение модулей разности полученных изображений}
	\label{tabular:timesandtenses}
	\begin{center}
		\begin{tabular}{ccc}
			Сигма итенсивности & Модуль разности \\
			10 & 14.699821190406311 \\
			20 & 14.699821190406306 \\
			30 & 14.699821190406302 \\
			40 & 14.699821190406311 \\
			60 & 14.699821190406308 \\
		\end{tabular}
	\end{center}
\end{table}

Вывод:

Наилучший результат наблюдается при $\sigma$ = 30.

\section{Вывод}
Результаты эксперимонтов показали, что лучшем фильтром является фильтр нелокальных средних. Но он имеет существенный недостаток - производительность.

\vfill % заполнить всё доступное ниже пространство

\section{Листинг}

\lstinputlisting[
label=code:equalization,
caption={Наложение шума},
]{../noise/main.py}
\parindent=1cm

\lstinputlisting[
	label=code:linearExtension,
	caption={Билотеральный метод},
]{../bilaterally/main(1).py}
\parindent=1cm

\lstinputlisting[
label=code:equalization,
caption={Метод Гауса},
]{../gaus/main.py}
\parindent=1cm

\lstinputlisting[
label=code:equalization,
caption={Метод нелокальных средних},
]{../nlm/main.py}
\parindent=1cm

\end{document}
